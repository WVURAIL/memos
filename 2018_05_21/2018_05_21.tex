%%% Research Diary - Entry
%%% Template by Mikhail Klassen, April 2013
%%% Modified by Pranav Sanghavi, May 2018
\documentclass[11pt,letterpaper]{article}

\newcommand{\workingDate}{\textsc{\today}}
\newcommand{\userName}{Pranav Sanghavi}



\usepackage{researchdiary_png}
\usepackage{tocbasic}

\begin{document}

{\Huge \today}\\[5mm] %Date of entry

\tableofcontents

\section{Interferometry}

The idea that the resolution of optical instruments is limited due to the wave nature of light is familiar to students of optics and is embodied in the Rayleigh's criterion which states that the angular resolution of a telescope/microscope is ultimately diffraction limited and is given by 

\begin{equation}	
	\theta \sim \lambda/D
\end{equation}

where $D$ is some measure of the aperture size. The need for higher angular resolution has led to the development of instruments with larger size and which operate at smaller wavelengths. In radioastronomy, the wavelengths are so large that even though the sizes of radio telescopes are large, the angular resolution is still poor compared to optical instruments. Thus while the human eye has a diffraction limit of $\sim~20^{''}$ and even modest optical telescopes have diffraction limits of $0.1^{''}$, even the largest radio telescopes (300m in diameter) have angular resolutions of only $\sim 10^{'}$ at 1 metre wavelength. To achieve higher resolutions one has to either increase the diameter of the telescope further (which is not practical) or decrease the observing wavelength. The second option has led to a tendency for radio telescopes to operate at centimetre and millimetre wavelengths, which leads to high angular resolutions. These telescopes are however restricted to studying sources that are bright at cm and mm wavelengths. To achieve high angular resolutions at metre wavelengths one need telescopes with apertures that are hundreds of kilometers in size. Single telescopes of this size are clearly impossible to build. Instead radio astronomers achieve such angular resolutions using a technique called aperture synthesis

\subsection{The Adding interferometer}

commodo consequat. Duis aute irure dolor in reprehenderit in voluptate velit esse cillum dolore eu fugiat nulla pariatur. Excepteur sint occaecat cupidatat non proident, sunt in culpa qui officia deserunt mollit anim id est laborum.

\begin{figure}[h]
\centering
\caption{} \label{fig:addinterferometer}
\end{figure}


\section{Correlator}

Lorem ipsum dolor sit amet, consectetur adipiscing elit, sed do eiusmod tempor incididunt ut labore et dolore magna aliqua. Ut enim ad minim veniam, quis nostrud exercitation ullamco laboris nisi ut aliquip ex ea commodo consequat. Duis aute irure dolor in reprehenderit in voluptate velit esse cillum dolore eu fugiat nulla pariatur. Excepteur sint occaecat cupidatat non proident, sunt in culpa qui officia deserunt mollit anim id est laborum.

Lorem ipsum dolor sit amet, consectetur adipiscing elit, sed do eiusmod tempor incididunt ut labore et dolore magna aliqua. Ut enim ad minim veniam, quis nostrud exercitation ullamco laboris nisi ut aliquip ex ea commodo consequat. Duis aute irure dolor in reprehenderit in voluptate velit esse cillum dolore eu fugiat nulla pariatur. Excepteur sint occaecat cupidatat non proident, sunt in culpa qui officia deserunt mollit anim id est laborum.


\end{document}